\documentclass{article}

\usepackage{tikz}
\usepackage[margin=1in, includefoot]{geometry}
\usepackage{amsmath}
\usepackage{xcolor}
\usepackage{fancyhdr}
\usepackage{listings}

\setlength\parindent{0pt}
\pagestyle{fancy}
\linespread{1}
\title{Matrix Calculus}
\date{}

\begin{document}

\maketitle

\section{Basics}
\subsection{Trace Concepts}
\begin{equation*}
  \text{tr}(A) = \sum_{i = 1}^{n} a_{ii}
\end{equation*}

If $A$ is $n$ x $m$, $B$ is $m$ x $n$, then
\begin{align*}
    AB &= [\sum_{k = 1}^{m} a_{ik} b_{kj}] \\
    \text{tr}(AB) &= \sum_{i = 1}^{n} \sum_{k = 1}^{m} a_{ik} b_{ki}
\end{align*}

If $A$ is $m$ x $n$, $B$ is $m$ x $n$, then
\begin{align*}
    A^{T}B &= [\sum_{k = 1}^{m} a_{ki} b_{kj}] \\
    \text{tr}(A^{T}B) &= \sum_{i = 1}^{n} \sum_{k = 1}^{m} a_{ki} b_{ki}
\end{align*}

\subsection{Trace Properties}
If $AB$ and $BA$ are both defined, then $A^{T}$ and $B$ have the same shape, and $\text{tr}(AB) = \text{tr}(BA)$, which is equal to $\sum_{i, j} a_{ij}b_{ji}$. \bigskip

Element-wise product (Hadamard product):  $tr(A^T (B \odot C)) = tr((A \odot B)^T B)$

\end{document}