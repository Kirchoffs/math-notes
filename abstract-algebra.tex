\documentclass{article}

\usepackage{tikz}
\usepackage[margin=1in, includefoot]{geometry}
\usepackage{amsmath}
\usepackage{xcolor}
\usepackage{fancyhdr}
\usepackage{listings}

\setlength\parindent{0pt}
\pagestyle{fancy}
\linespread{1}
\title{Abstract Algebra}
\author{}
\date{}

\begin{document}
\maketitle

\section{Preliminaries}
\subsection{Set Theory}
\subsubsection{Function \& Map}
\textbf{Concepts:} \par
$$ A \xrightarrow[]{f} B $$
It denotes a function $f$ from $A$ to $B$ and the value of $f$ at a is denoted at $f(a)$. 
The set $A$ is called the domain of $f$ and $B$ is called the codomain of $f$. \bigskip

$$ f(A) = \{b \in B \, | \, b = f(a), for \; a \in A\} $$
The set $f(A)$ is a subset of $B$, called the range or image of $f$. \bigskip

$$ f^{-1}(C) = \{ a \in A \, | \, f(a) \in C \} $$
For each subset $C$ of $B$, the set $f^{-1}(C)$ consisting elements of $A$ mapping into $C$ under $f$ is called the preimage
or inverse image of $C$. \
For each $ b \in B $, the preimage of $\{b\}$ under $f$ is called the fiber of $b$ over $f$. \bigskip

$$ (g \circ f)(a) = g(f(a)) $$
if $ f: A \rightarrow B $ and $ g: B \rightarrow C $, then the composite map $ g \circ f: A \rightarrow C $ is defined by above equation. \bigskip

$$ f: A \rightarrow B $$
\begin{itemize}
    \item $f$ is injective or is an injection if whenever $ a_{1} \ne a_{2} $, then $ f(a_{1}) \ne f(a_{2}) $
    \item $f$ is surjective or is a surjection if for all $ b \in B $ there is some $ a \in A $ 
          such that $ f(a) = b $, i.e., the image of f is all of B.
    \item $f$ is bijective or is a bijection if it is both injective and surjective. If such a bijection $f$
          exists from $A$ to $B$, then we say $A$ and $B$ are in bijective correspondence.
    \item $f$ has a left inverse if there is a function $ g: B \rightarrow A $ such that $ g \circ f: A \rightarrow A $ is the identity map on A,
          i.e., $ (g \circ f)(a) = a $, for all $ a \in A $.
    \item $f$ has a right inverse if there is a function $ g: B \rightarrow A $ such that $ f \circ g: B \rightarrow B $ is the identity map on B,
    i.e., $ (f \circ g)(b) = b $, for all $ b \in B $.
\end{itemize} \bigskip

\textbf{Proposition:} \par

\end{document}
