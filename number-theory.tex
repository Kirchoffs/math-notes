\documentclass{article}

\usepackage{tikz}
\usepackage[margin=1in, includefoot]{geometry}
\usepackage{amsmath}
\usepackage{xcolor}
\usepackage{fancyhdr}
\usepackage{listings}

\setlength\parindent{0pt}
\pagestyle{fancy}
\linespread{1}
\title{Number Theory}
\date{}

\begin{document}

\maketitle

\section{Greatest Common Divisors}

\section{Congruence}
  \subsection{Modular Inverse}
    A modular inverse of an integer b (modulo p) is the integer $ b^{-1} $ such that:
      $$ b \, b^{-1} \equiv 1 \; mod \; p $$
    \subsubsection{Extended Euclidean Method}
      if $ b \, b^{-1} \equiv 1 \; mod \; p $, then we have:
      \begin{align*}
        b \, b^{-1} = py + 1 \\
        b \, b^{-1} - py = 1 
      \end{align*}
      If we can solve the indeterminate equation $ bx + py = 1 $, then we can get the value of $ b^{-1} $, which is $ x $. \\
      To get the $ x $ value, we can use extended euclidean method.
      \begin{lstlisting}[language=Python, basicstyle=\small]
        def extended_gcd(a, b):
            if b == 0:
                return (1, 0)
            x, y = extended_gcd(b, a % b)
            return (y, x - a // b * y)
        
        def get_modular_inverse(x, p):
            res, _ = extended_gcd(x, p)
            while res < 0 {
                res += p
            }
            return res
      \end{lstlisting}

  \subsection{Lucas Theorem}
    For non-negative integers m and n and a prime p, the following congruence relation holds:
    \begin{equation*}
      \binom{m}{n} \equiv \prod_{i = 0}^{k} \binom{m_i}{n_i} \; mod \; p
    \end{equation*}
    
    Where
    \begin{align*}
      m &= m_{k}p^{k} + m_{k-1}p^{k-1} + ... + m_{1}p + m_{0} \\
      n &= n_{k}p^{k} + n_{k-1}p^{k-1} + ... + n_{1}p + n_{0}
    \end{align*}

    Another form:
    \begin{equation*}
      \binom{m}{n} \equiv \binom{\lfloor m/p \rfloor}{\lfloor n/p \rfloor} \binom{m \; mod \; p}{n \; mod \; p} \; mod \; p
    \end{equation*}
  
  \subsection{Common Equation and Conclusions}
    if $ a_1 \equiv b_1 \; mod \; m $, $ a_2 \equiv b_2 \; mod \; m $ then:
    \begin{align*}
        a_1 + a_2 & \equiv b_1 + b_2 \; mod \; m \\
        a_1 - a_2 & \equiv b_1 - b_2 \; mod \; m \\
        a_1 \times a_2 & \equiv b_1 \times b_2 \; mod \; m
    \end{align*}

    if $ a \equiv b \; mod \; m $, $ k > 0 $, and $d$ is a common divisor of $a$, $b$ and $m$, then:
    \begin{align*}
      ak & \equiv bk \; mod \; m \\
      \frac{a}{d} & \equiv \frac{b}{d} \; mod \; \frac{m}{d}
    \end{align*}
  
\section{Numeral System}
  \subsection{Number of bits for a number N with base b}
    Assume there are k bits, then:
    \begin{align*}
      b^{k-1} \leq& \;N < b^k \\
      log_b^N <& \;k \leq log_b^{N} + 1
    \end{align*}
    If such integer k does not exist, then we must have an integer $t$:
    \begin{equation*}
      \begin{cases}
        t \leq log_b^N \\
        log_b^N + 1 < t + 1
      \end{cases}
    \end{equation*}
    Under this condition, we have:
    \begin{equation*}
      \begin{cases}
        N >= b^t \\
        N < b^t
      \end{cases}
    \end{equation*}
    which is impossible. So we can conclude that $k$ must exist, and the value of $k$ is $\lfloor log_b^N + 1 \rfloor$ or $\lceil log_b^{N + 1} \rceil$.
    The latter one is calculated by:
    \begin{align*}
      N \leq& \;b^k - 1 \\
      k \geq& \;log_b^{N + 1} \\
    \end{align*}
    To represent N numbers from 0 to N - 1, we need $\lfloor log_b^{N-1} + 1 \rfloor$ or $\lceil log_b^N \rceil$ bits. \\
    Similarly, for a binary tree of N nodes, the min height (if a single node has height 1) is $\lfloor log_b^N + 1 \rfloor$ or $\lceil log_b^{N + 1} \rceil$.

\section{Others}
  \subsection{Gray Code}
    \subsubsection{Transform}
      For a n bits number, the index of digits from right to left is 0 to n - 1. The number can be represented as
      $B_{n-1}...B_{1}B_{0}$.
      \begin{equation*}
        \begin{cases}
          G_{n-1} = B_{n-1} \\
          G_{i} = B_{i} \oplus B_{i+1}, 0 \leq i \leq n - 2
        \end{cases}
      \end{equation*}

      \begin{equation*}
        \begin{cases}
          B_{n-1} = G_{n-1} \\
          B_{i} = G_{i} \oplus B_{i+1}, 0 \leq i \leq n - 2
        \end{cases}
      \end{equation*}

    \subsubsection{Construction}
      \textbf{Method 1:} \par
      If we have two digits gray code 00 01 11 10, to construct three digits gray code,
      we can make a mirror symmetry: 00 01 11 10 \vline \;10 11 01 00.

      \smallskip 

      Then for the first part we append the prefix 0, for the second part, we append 1. finally, we have: \par
      000 001 011 010 \vline \;110 111 101 100.

      \bigskip

      \textbf{Method 2:} \par
      From the starting all-zero gray code, we can construct the next two numbers using following two steps:

      \smallskip
      
      (1). Flip the least significant digit to get the next gray code. \par
      (2). Flip the left bit of the rightmost 1 to get the next gray code. 
    
    \subsubsection{Formula}
      \begin{equation*}
        G(n) = n \oplus \lfloor{\frac{n}{2}}\rfloor
      \end{equation*}
\end{document}