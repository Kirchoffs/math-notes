\documentclass{article}

\usepackage{tikz}
\usepackage[margin=1in, includefoot]{geometry}
\usepackage{amsmath}
\usepackage{xcolor}
\usepackage{fancyhdr}
\usepackage{listings}

\setlength\parindent{0pt}
\pagestyle{fancy}
\linespread{1}
\title{Combinatorics}
\author{}
\date{}

\begin{document}

\maketitle

\section{Permutation and Combination}
  \subsection{Multinomial Theorem}
    \begin{equation*}
      (x_1 + x_2 + ... + x_m)^n = 
      \sum_{k_1 + k_2 + ... + k_m = n; \: k_1, k_2, ..., k_m \,\geq\, 0}
      \binom{n}{k_1, k_2, ..., k_m} \prod_{t = 1}^{m} x_t^{k_t}
    \end{equation*}

    where
    \begin{equation*}
      \binom{n}{k_1, k_2, ..., k_m} = \frac{n!}{k_1! k_2! ... k_m!}
    \end{equation*}

    \bigskip
    
    When $m = 2$, we have binomial theorem:
    
    \begin{align*}
      (x_1 + x_2)^n & = \sum_{k = 0}^{n} \binom{n}{k} x^k \\
                    & = \sum_{k = 0}^{n} \frac{n!}{k! \: (n-k)!} x^k \\
                    & = \sum_{k = 0}^{n} \binom{n}{k, \: n-k} x^k \\
    \end{align*}

    \bigskip

    In python, we can use the following code to calculate the multinomial coefficient:
    \begin{lstlisting}[language=Python, basicstyle=\small]
      from math import factorial, prod
      def multinomial(arr):
          return (
              factorial(sum(arr)) // 
              prod([factorial(num) for num in arr])
          )
    \end{lstlisting}
  \subsection{Lucas Theorem}
\end{document}